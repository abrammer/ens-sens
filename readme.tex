\documentclass[psfig,12pt]{article}

\usepackage{graphicx}
\usepackage{amssymb}
\usepackage[T1]{fontenc}
\usepackage{times}

\setlength{\topmargin}{0pt}
\setlength{\headheight}{0pt} % *** for xdvi and ps ***
\setlength{\headsep}{0pt}
\setlength{\textheight}{9.0truein}
\setlength{\textwidth}{6.5truein}
\setlength{\footskip}{24pt}
\setlength{\oddsidemargin}{0pt}
\setlength{\marginparsep}{2pt}
\setlength{\marginparwidth}{5pt}
\setlength{\parskip}{1em}
\hyphenpenalty=2000
\renewcommand{\baselinestretch}{1.0}

%%%%% Start document %%%%%
\begin{document}
\pagestyle{empty}

\centering
{\Large\bf NHC Ensemble-based Sensitivity Code Readme}
\flushleft
\vspace{0.3in}

This document provides a description of the enclosed code, that can be used to compute ensemble-based
sensitivity from gridded ensemble forecast data.  The enclosed set of code is written in conda python
and is designed to work with any grib files that contain forecast data on a lat/lon grid.  Most of the
settings for this program are set within a configuration/parameter file, while the date, storm, and
parameter file itself are set at the command line.  Furthermore,
the code is designed to work for a variety of models and computing locations.  Those differences related
to various locations and models is also isolated to the individual i/o module, which has common routine
names, but does all of the model/location specific differences inside of it and is transparent to 
the rest of the code.

The code itself consists of four distinct stages, all of which are controled by run\_NHC\_sens.py

\begin{enumerate}
\item Data staging and preparation (model and platform specific)
\item Computing forecast metrics (fcst\_metrics\_tc.py)
\item Compute forecast fields to compute sensitivity to (compute\_tc\_fields.py)
\item Compute sensitivity and generate maps (nhc\_sens.py)
\end{enumerate}

In order to generate sensitivity output, the user should run the following code from the unix 
command line, which is the command for the Hurricane Laura forecast initialized 0000~UTC 22 August 2020:

\vspace{0.1in}
python run\_NHC\_sens.py --init 2020082200 --storm laura13l --param nhc\_ecmwf.py, 
\vspace{0.1in}

where the --init argument is the forecast initialization date in yyyymmddhh format, --storm is the 
TC name, including both the name, TC number, and the basin.  The TC number and basin are necessary 
as the code parses the storm text string to figure out the basin and TC number.  Finally, --param is
the path to the paramter/configuration file, that contains a number of configuration options that are
meant to be static from one initialization time/storm to another, but still gives the user the option
to change how the code executes, or how the plots look.  Most of the configuration options have 
default values, though some MUST be set within the file for the code to work.  The tables below list 
the individual parameter/configuration options available and the default values, where appropriate.

The outcome of running this code is a set of output directories that include the graphical and 
gridded sensitivity output.  The format of these directory is:

\vspace{0.1in}
\{figure\_dir\}/\{storm\}\_\{yyyymmddhh\}/\{metric\}/sens/\{field\},
\vspace{0.1in}

where storm is the name of the TC (same as --storm line above), yyyymmddhh is the initialization date
(same as --init line above), metric is the name of each forecast metric, where each forecast metric has its own
directory.  For example, the integrated track metric (the default metric of the code) is named 
f120\_intmajtrack.  Positive values of the metric are indicative of a TC that will end up further along and/or i
to the right of the ensemble-mean track.  The advantage of using this position metric is that it does not 
require specifying a particular lead time and takes into account the temporal correlation of forecast tracks 
(i.e., members that are further west early in the forecast will end up further west later on).  The user
can specify additional metrics to compute sensitivity for using the metrics configuration option.

Within each forecast metric directory are two sub-directories, one is called sens,
which are the sensitivity plots/grids on a fixed domain, while the other are the plots/grids on a storm-centered
grid.  Within each of these directories is a set of subdirectories that represent individual
forecast fields that you are computing the sensitivity to.  The forecast hour in each file's name is the
forecast lead time that you are computing the sensitivity to (i.e., a file starting with 202008200\_f036 is 
the sensitivity of the metric to the 36~h forecast fields.).  The table below gives the list of fields
for which the sensitivity of TC track/position forecasts are computed and what they represent:

\begin{table}[H]
\begin{center}
\begin{tabular}{|p{1.25in}|p{5.0in}|}
\hline
Parameter Name & Description \\  \hline\hline
usteer & Zonal component of the steering flow.  By default, this is designated as the average wind between
300-850~hPa (vortex removed), but this can be changed in the configuration file. \\ \hline
vsteer & Meridional component of the steering flow.  By default, this is designated as the average wind between
300-850~hPa (vortex removed), but this can be changed in the configuration file. \\ \hline
masteer & Major axis winds are the wind component that is in the direction of greatest track variability for 
that particular case (positive values are either along and/or right of track).  In most situations, the 
sensitivity to the major axis wind is the most useful for sensitivity calculations because it most closely 
relates to variability in subsequent TC position, which is not often in the Cartesian directions.  \\ \hline
pv250hPa & 250 hPa potential vorticity \\ \hline
h500 & 500 hPa height \\ \hline
\end{tabular}
\end{center}
\end{table}

\begin{table}[H]
\caption{Configuration options for the model subset.}
\begin{center}
\begin{tabular}{|p{1.75in}|p{0.5in}|p{4.00in}|}
\hline
Parameter Name & Type & Description \\  \hline\hline
model\_src & string & Name of the model being used in the sensitivity calculation.  This is mainly used in
plot titles, so the user can set to whatever they want.  No default value.  \\  \hline
io\_module & string & Name of the module to use for obtaining and reading the grib and ATCF file.  Each
platform and model will have its own module.  This value MUST be set by the user. \\  \hline
num\_ens & integer & Number of perturbation ensemble members (i.e., ECMWF has 50 perturbed members, 
GEFS has 30).  No default value, so it must be set. \\  \hline
fcst\_hour\_int & integer & Forecast hour interval for computing forecast fields for sensitivity 
calculations in hours.  Default:  12 h \\  \hline
fcst\_hour\_max & integer & Last forecast hour to compute forecast fields for sensitvity 
calculations in hours.  Default 120 h \\  \hline
tigge\_forecast\_time & string & List of forecast hours to read from ECMWF TIGGE archive. \\ \hline
tigge\_forecast\_grid\_space & string & Requested grid spacing for data pulled from the 
TIGGE respository.  Default: 1.0/1.0 (one degree) \\ \hline
flip\_lon & boolean & Swap longitude from -180 to 180 to 0 to 360.  Default False \\ \hline
\end{tabular}
\end{center}
\end{table}

\begin{table}[H]
\caption{Configuration options for the local subset.}
\begin{center}
\begin{tabular}{|p{1.60in}|p{0.5in}|p{4.15in}|}
\hline
Parameter Name & Type & Description \\ \hline\hline
atcf\_dir & string & Path to raw ATCF forecast data on local server.  No default value \\ \hline
model\_dir & string & Path to raw model data on local server.  No default value \\ \hline
best\_dir & string & Path to best track data on local server.  No default value \\ \hline
work\_dir & string & Path to work directory where sensitivity calculations are carried out
No default value. \\ \hline
output\_dir & string & Path to directory to save certain output of the sensitivity 
calculations, if desired.  No default value.  \\ \hline
script\_dir & string & Path to python scripts and modules (i.e., where this code is located.)
No default value. \\ \hline
figure\_dir & string & Path to directory where output figures will be placed.
No default value.  \\ \hline
outgrid\_dir & string & Path to directory where gridded sensitivity output will be placed.
No default value.  \\ \hline
log\_dir & string & Path to directory where log file output from the python logging function
will be placed.  No default value. \\ \hline
archive\_metric & boolean & True to save metric netcdf files into the appropriate 
output\_dir.  Default: False \\ \hline
archive\_fields & boolean & True to save forecast field netcdf files into the appropriate
output\_dir.  Default: False \\ \hline
save\_work\_dir & boolean & True to save work directory at the end of the execution.
Default:  False \\ \hline

\end{tabular}
\end{center}
\end{table}

\begin{table}[H]  
\caption{Configuration options for the vitals\_plot subset.}
\begin{center}
\begin{tabular}{|p{1.60in}|p{0.5in}|p{4.15in}|}
\hline
Parameter Name & Type & Description \\ \hline \hline
trackfile & string & Name of figure that shows the TC track forecast.  Default:  track.png \\ \hline
track\_output\_dir & string & Name of directory to place the track plot.  Default: figure\_dir from above \\ \hline
intfile & string & Name of figure that shows the TC intensity forecast.  Default:  intensity.png \\ \hline
int\_output\_dir & string & Name of directory to place the intensity plot.  Default: figure\_dir from above \\ \hline
forecast\_hour\_int & float & Frequency of forecasts from model (hours).  Default: 6 \\ \hline
forecast\_hour\_max & string & Maximum forecast hour for track and intensity plots.  
Default:  120 hours \\ \hline
plot\_ellipse & boolean & True to plot TC position ellipses on plots.  Default:  True \\ \hline
ellipse\_frequency & float & Frequency to plot the position ellipses (hours).  Default: 24 \\ \hline
plot\_best & boolean & True to plot best track information, if available.  Default:  True \\ \hline
projection & string & Map projection to use for vitals plot.  Default:  PlateCarree \\ \hline
grid\_interval & float & Latitude and Longitude line grid interval (degrees).  Default:  5 \\ \hline
\end{tabular}
\end{center}
\end{table}

\begin{table}[H]
\caption{Configuration options for the metric subset.}
\begin{center}
\begin{tabular}{|p{1.75in}|p{0.5in}|p{4.00in}|}
\hline
Parameter Name & Type & Description \\ \hline \hline
metric\_hours & float & Vector list of forecast hours to compute forecast metrics.  No default value \\ \hline
track\_eof\_hour\_init & float & Initial forecast hour to use for integrated track metric (hours).  Default: 24 \\ \hline
track\_eof\_hour\_int & float & Forecast hour interval to use for integrated track metric (hours).  Default:  6 \\ \hline
track\_eof\_hour\_final & float & Final forecast hour to use for integrated track metric (hours).  Default: 120 \\ \hline
track\_eof\_member\_frac & float & Fraction of members that need to be present during a forecast hour for track EOF.  Default:  0.3 \\ \hline
track\_eof\_esign & float & Factor to multiply EOF PC by.  Default 1.0 \\ \hline
intensity\_eof\_hour\_init & float & Initial forecast hour to use for integrated intensity metric (hours).  Default: 24 \\ \hline
intensity\_eof\_hour\_int & float & Forecast hour interval to use for integrated intensity metric (hours).  Default:  6 \\ \hline
intensity\_eof\_hour\_final & float & Final forecast hour to use for integrated intensity metric (hours).  Default: 96 \\ \hline
intensity\_eof\_member\_frac & float & Fraction of members that need to be present during a forecast hour for intensity EOF.  Default:  0.3 \\ \hline
kinetic\_energy\_metric & boolean & True to calculate area-average kinetic energy metric.  Default:  False \\ \hline
kinetic\_energy\_radius & float & Radius over which to calculate the average kinetic energy (km).  Default:  200 \\ \hline
kinetic\_energy\_level & float & Pressure level to use for kinetic energy metric (hPa).  Default: 1000 \\ \hline
wind\_speed\_eof\_metric & boolean & True to calculate forecast metric that is EOF/PC of the
wind speed over the specified time window and area.  Currently experimental.  Default:  False \\ \hline
wind\_metric\_file & string & Path to file that contains precipitation EOF metric settings.
Assumes the file has the format \{yyyymmddhh\}\_\{storm\}\_wind.  No default value. \\ \hline
precipitation\_metric & boolean & True to calculate forecast metric that is the average
precipitation over the specified time window and area.  Currently experimental.  Default:  False \\ \hline
precipitation\_eof\_metric & boolean & True to calculate forecast metric that is EOF/PC of the 
precipitation over the specified time window and area.  Currently experimental.  Default:  False \\ \hline
precip\_metric\_file & string & Path to file that contains precipitation EOF metric settings.  
Assumes the file has the format \{yyyymmddhh\}\_\{storm\}\_precip.  No default value. \\ \hline
projection & string & Map projection to use for metric plot.  Default:  PlateCarree \\ \hline
grid\_interval & float & Latitude and Longitude line grid interval (degrees).  Default:  5 \\ \hline
\end{tabular}
\end{center}
\end{table}

\begin{table}[H]
\caption{Configuration options for the field subset.}
\begin{center}
\begin{tabular}{|p{1.60in}|p{0.5in}|p{4.15in}|}
\hline
Parameter Name & Type & Description \\ \hline \hline
calc\_uvsteer & boolean & True to compute the u/v component of the steering wind. Default: True \\ \hline
calc\_steer\_circ & boolean & True to compute the circulation/vorticity from the steering wind.  Default:  False \\ \hline
steer\_level1 & float & Lowest pressure level to use to compute the layer-average steering wind (hPa).  Default: 300 \\ \hline
steer\_level2 & float & Highest pressure level to use to compute the layer-average steering wind (hPa).  Default: 850 \\ \hline
steer\_radius & float & Radius to use for removing the TC from the model fields (km).  Default: 333 \\ \hline
calc\_height & boolean & True to compute gepotential height on specified pressure levels.  Default:  True \\ \hline
height\_levels & float & List of pressure levels to compute the geopotential height (hPa).
Default:  500 \\ \hline
calc\_pv250hPa & boolean & True to compute PV on the 250 hPa surface.  Default:  True \\ \hline
calc\_theta-e & boolean & True to compute equivalent potential temperature at specified pressure levels.
Default:  False \\ \hline
theta-e\_levels & float & List of pressure levels to compute equivalent potential temperature (hPa).
Default:  700, 850 \\ \hline
calc\_q500-850hPa & boolean & True to compute the integrated water vapor between 500 and 
850~hPa.  Default:  False \\ \hline
calc\_winds & boolean & True to compute zonal and meridional wind at specified pressure levels.
Default:  False \\ \hline
wind\_levels & float & List of pressure levels to compute zonal and meridional wind (hPa).
Default:  850 \\ \hline
min\_lat & float & Minimum latitude to compute forecast fields over.  Default:  0.0 \\ \hline
max\_lat & float & Maximum latitude to compute forecast fields over.  Default:  65.0 \\ \hline
min\_lon & float & Minimum longitude to compute forecast fields over.  Default:  -180.0 \\ \hline
max\_lon & float & Maximum longitude to compute forecast fields over.  Default:  -10.0 \\ \hline
\end{tabular}
\end{center}
\end{table}

\begin{table}[H]
\caption{Configuration options for the sens subset.  This set of parameters set the display options
for the sensitivity plots.}
\begin{center}
\begin{tabular}{|p{1.60in}|p{0.5in}|p{4.15in}|}
\hline
Parameter Name & Type & Description \\  \hline \hline
metrics & string & List of names of forecast metrics to compute the sensitivity to.  Default:  none \\ \hline
min\_lat & float & Minimum latitude for sensitivity plots.  Default:  8.0, also set in
run\_NHC\_sens.py based on basin.  \\ \hline
max\_lat & float & Maximum latitude for sensitivity plots.  Default:  65.0, also set in
run\_NHC\_sens.py based on basin.  \\ \hline
min\_lon & float & Minimum longitude for sensitivity plots.  Default:  -140.0, also set in 
run\_NHC\_sens.py based on basin.  \\ \hline
max\_lon & float & Maximum longitude for sensitivity plots.  Default:  -20.0, also set in
run\_NHC\_sens.py based on basin.  \\ \hline
zero\_non\_sig\_sens & boolean & True to plot only the statistically significant 
sensitivity locations.  Default:  False   \\ \hline
grid\_interval & float & Latitude and Longitude line grid interval (degrees).  Default:  10$^{\circ}$.  \\ \hline
barb\_interval & integer & Number of grid points in between each wind barb in the plot.
Default:  6 grid points  \\ \hline
dropsonde\_file & string & Full path to file of dropsonde locations.  Default:  none \\ \hline
drop\_file\_type & string & Type of dropsonde file to read. Default: nhc \\ \hline
drop\_mark\_size & integer &  Marker size of dropsonde locations in plot.  Default:  6 \\ \hline
drop\_mark\_color & string & Dropsonde marker color in plot.  Default: black \\ \hline
drop\_mark\_type & string & Dropsonde marker in plot.  Default: + \\ \hline
rawinsonde\_file & string & Full path to file of rawinsonde locations.  Default:  none \\ \hline
rawin\_mark\_size & integer & Marker size of rawinsonde locations in plot.  Default:  6 \\ \hline
rawin\_mark\_color & string & Rawinsonde marker color in plot.  Default: gray \\ \hline
range\_rings & boolean & True to plot range rings from the predicted TC center.  Default: True \\ \hline
ring\_values & floats & List of range ring radii for plot in km.  Default:  \\ \hline
output\_sens & boolean & True to create netCDF file that contains gridded sensitivity 
fields that can be used in AWIPS or traveling salesman.  Default:  True \\ \hline
nhc\_sens & boolean & True to create NHC version of the gridded netCDF file, which means that
it includes a variable that is the absolute value of sensitivity (for traveling salesman software).
Default:  False \\ \hline
\end{tabular}
\end{center}
\end{table}

\end{document}
